%!TEX root = ../main.tex
\documentclass[../main]{subfiles}

\begin{document}

\newpage
\phantomsection
\addcontentsline{toc}{section}{Заключение}
\section*{Заключение}

Исходя из анализа методов формализации политик безопасности, было принято решение продолжать движение в сторону создания инструментов разметки датасетов, и моделей глубокого обучения. Таким образом было проведено первичное планирование процесса выполнения выпускной квалификационной работы магистра.

В результате выполнения работы было спроектировано и реализовано требуемое программное средство для сбора датасета, ориентированного на политики безопасности, и позволяющего создавать, обучающие выборки, ориентированные на формирование онтологического представления предметной области.

В ходе выпускной квалификационной работы были успешно проделаны следующие шаги:

\begin{itemize}
    \item провести анализ предметной области;
    \item разработать методики сбора, очистки и разметки обучающей выборки;
    \item спроектировать инструментарий для построения обучающей выборки, обеспечивающей обучение классификатора с учетом онтологического представления предметной области;
    \item реализовать  инструментарий для построения обучающей выборки, обеспечивающей обучение классификатора с учетом онтологического представления предметной области.
\end{itemize}

Все задачи, поставленные в выпускной квалификационной работе, были успешно выполнены. Файлы исходных кодов приложения приведены в приложении \hyperref[sec:appendix]{A}.  Электронная  версия  данной  пояснительной  записки  к выпускной квалификационной работе представлена в приложении \hyperref[sec:appendix]{A}.


\end{document}