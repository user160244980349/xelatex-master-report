%!TEX root = ../main.tex
\documentclass[../main]{subfiles}

\begin{document}

\newpage
\phantomsection
\addcontentsline{toc}{section}{Введение}
\section*{Введение}

В настоящее время персональные данные широко используются в пре\-доставлении цифровых услуг, их персонализации и улучшении. Персональные данные -- это любые данные, идентифицировать физическое лицо \cite{GDPR}. Таким образом, личные данные -- это не только биометрическая информация, данные о состоянии здоровья человека, а также фото абонента услуги, местонахождение, информация о приложении и устройстве, которое можно использовать для отслеживания действий и информации о потребителе. Несколько массовых утечек персональных данных за последнее десятилетие привело к ужесточению законодательных требований во многих странах по всему миру. В настоящее время требуется, чтобы все личные данные обрабатывались надежно, а действия с ними были ясны и прозрачны для субъекта данных в соответствии с его или ее явно указанным согласием. Политики конфиденциальности поставщиков услуг, онлайн-согласие пользователей -- единственные законные документы, сообщающие конечным пользователям, как собираются, обрабатываются их личные данные и передается третьим лицам. Однако в большинстве случаев эти документы написаны так, что их довольно сложно понять. И в настоящее время ситуация такова, что законодательные требования  соблюдаются производителями продукции и поставщиков услуг, но конечные пользователи дают свое согласие без четкого понимания того, как обрабатываются их личные данные, потому что политика конфиденциальности и онлайн-согласие пользователя читаются редко из-за их сложности и низкой читабельности. Это ведет к ситуациям, когда конечные пользователи не знают о рисках для конфиденциальности связанных с использованием определенной услуги или устройства. 

На момент написания выпускной квалификационной работы актуальность данной работы является высокой, так как формализация политик безопасности открывает возможности для более простой и ясной формулировки этих, что уменьшит количество угроз персональным данным. Также становится возможной разрабатка методик расчета рисков потребления цифровых услуг и устройств.

Цель работы -- разработать эффективный план автоматизированных решений для формализации политик безопасности на основе онтологического представления, разработать инструменты создания обучающей выборки для автоматизированной формализации политик безопасности. 

В ходе выполнения предполагается реализация инструментов для сбора датасета, который будет применен для обучения классификатора. Классификатор позволит автоматизированно формализовать политики безопасности. По формализованному описанию политик станет возможной оценка рисков для персональных данных пользователей.

Для достижения данной цели необходимо:

\begin{itemize}
    \item провести анализ предметной области;
    \item разработать методики сбора, очистки и разметки обучающей выборки;
    \item спроектировать инструментарий для построения обучающей выборки, обеспечивающей обучение классификатора с учетом онтологического представления предметной области;
    \item реализовать  инструментарий для построения обучающей выборки, обеспечивающей обучение классификатора с учетом онтологического представления предметной области.
\end{itemize}

Выпускная  квалификационная  работа  состоит  из  введения, четырех разделов и заключения. В первом разделе производится анализ предметной области. Во втором разделе проведены экспериметнты со строгими методами текстового анализа и обоснование необходимости использования моделей текстового анализа, основанных на глубоком обучении. В третьем разделе описаны приемы и методики проектирования, аргументация их применения. В четвертом разделе описан процесс разработки и полученные результаты. В пятом разделе предложен план по коммерциализации научно-исследовательской работы магистра.

\end{document}