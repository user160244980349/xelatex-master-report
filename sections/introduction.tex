%!TEX root = ../main.tex
\documentclass[../main]{subfiles}

\begin{document}

\newpage
\phantomsection
\addcontentsline{toc}{section}{Введение}
\section*{Введение}

В настоящее время персональные данные широко используются в предоставлении цифровых услуг, их персонализации и улучшении. Персональные данные -- это любые данные, идентифицировать физическое лицо \cite{GDPR}. Таким образом, личные данные -- это не только биометрическая информация, данные о состоянии здоровья человека, а также фото абонента услуги, местонахождение, информация о приложении и устройстве, которое можно использовать для отслеживания действий и информации о потребителе. Несколько массовых утечек персональных данных за последнее десятилетие привело к ужесточению законодательных требований во многих страны по всему миру. В настоящее время требуется, чтобы все личные данные обрабатывались надежно, а действия с ними были ясны и прозрачны для субъекта данных в соответствии с его или ее явно указанным согласием. Политики конфиденциальности поставщиков услуг, онлайн-согласие пользователей -- единственные законные документы, сообщающие конечным пользователям, как собираются, обрабатываются их личные данные и передается третьим лицам. Однако в большинстве случаев эти документы написаны так, что их довольно сложно понять. И в настоящее время ситуация такова, что законодательные требования  соблюдаются производителями продукции и поставщиков услуг, но конечные пользователи дают свое согласие без четкого понимания того, как обрабатываются их личные данные, потому что политика конфиденциальности и онлайн-согласие пользователя читаются редко из-за их сложности и низкой читабельности. Это ведет к ситуациям, когда конечные пользователи не знают о рисках для конфиденциальности связанных с использованием определенной услуги или устройства. 

В настоящее время сфера информационных технологий является одной из самых быстрорастущих, в ней решается множество задач прикладного характера. Одной из прогрессивных технологий является технология глубокого обучения. Полученные с помощью глубокого обучения модели способны решать широкий спектр прикладных задач. Однако, у данного подхода имеется существенный недостаток -- необходимость датасета для обучения. Датасет играет критически важную роль в формировании результата в целом. Если качество датасета будет посредственным, либо он окажется недостаточно объемным, то поставленная задача не будет решена с адекватной точностью. В то же время сбор датасета -- работа кропотливая и рутинная. Отличным решением является автоматизация данного процесса, возможно не полная, но частичная. Она абсолютно точно увеличит скорость сбора информации, что позволит за то же время собирать более объемные датасеты, и как следствие более точные модели будут получены после обучения.

Целью выпускной квалификационной работы является разработка программного пакета для автоматизированной формализации политик безопасности с последующей оценкой рисков для персональных данных пользователей сервисов. В ходе выполнения предполагается реализация инструментов для сбора датасета, который будет применен для обучения классификатора. Классификатор позволит автоматизированно формализовать политики безопасности. По формализованному описанию политик станет возможной оценка рисков для персональных данных пользователей.

\end{document}