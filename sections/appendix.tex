%!TEX root = ../main.tex
\documentclass[../main]{subfiles}

\begin{document}

\newpage

\section*{Приложение А}
\phantomsection
\addcontentsline{toc}{section}{Приложение А}

\phantomsection
\label{sec:appendix11}
\subsection*{А.1 Алгоритм}

\begin{lstlisting}
coeff = getSumPdWeight(PD_class)
if coeff < 1:
    coeff = 10 + coeff
RiskScoreBase = max_criticality * (1 + lg(coeff))
\end{lstlisting}


\phantomsection
\label{sec:appendix12}
\subsection*{А.2 Алгоритм}

\begin{lstlisting}
if (LB_weight + P_weight) < 1:
    coeff = 1 
else:
    coeff = 1 + lg(LB_weight + P_weight)
FP_RiskScore = RiskScoreBase * coeff
if FP_RiskScore > 10:
    FP_RiskScore = RiskScoreBase + (10 - RiskScoreBase) * \ 
        (coeff / (1 + lg(6))
Return FP_RiskScore
\end{lstlisting}

\phantomsection
\label{sec:appendix13}
\subsection*{А.3 Алгоритм}

\begin{lstlisting}
UsageScenarioRiskScore = RiskScoreBase * coeff,

where UsageScenarioRiskScore – privacy risk score for the data usage scenario;
       RiskScoreBase – the base of privacy risk score;
       coeff – coefficient that increase or decrease the risk depending on the aspects specified in the privacy policy.

The algorithm’s pseudocode is provided below.

rC = getRoots(C) //C is the set of ontology classes
for rci from rC:  //rC is the set of root ontology classes corresponding 
                                                  to the usage scenarios
    coeff = 0    //coefficient that depends on the aspects specified 
                                                  in the privacy policy
    rcc = getChilds(rci)       //get childs of the selected root class
    for rcck from rcc:
        scc = getSubclasses(rcck)  //get subclasses of the selected 
                                                                  class
        for sccr from scc:
            catr = getCategory(sccr)  //determine category of the 
                                                               subclass
            add catr to cat  //forming the set of categories for the 
                                                               subclass
        maxCritCat = getMaxCritCat(cat)  //define category with max 
                                                            criticality
        classWeight = getClassWeight(maxCritCat)  //define weight of 
                                       the category with max criticality
        if rcck is PersonalData:
            pd_coeff = getSumPdWeight(cat)  //calculate sum of weights 
                                            for personal data subclasses
            if pd_coeff < 1: 
                pd_coeff = 10 + pd_coeff
            RiskScoreBase = maxCritCat(1 + lg(pd_coeff))  //calculate 
                                         the base of privacy risk score
        else:
            coeff += classWeight  //calculate sum of max weights for 
                         other categories subclasses (not personal data)
        if coeff < 1: 
            risk_coeff = 1 
        else:
            risk_coeff = 1 + lg(coeff) 
    UsageScenarioRiskScore = RiskScoreBase * risk_coeff
    if UsageScenarioRiskScore > 10:       //scaling to 0 to 10 scale
        UsageScenarioRiskScore = RiskScoreBase + (10 - RiskScoreBase) * risk_coeff / (1 + lg(coeff)) 
add UsageScenarioRiskScore to RiskScores  //set of privacy risk scores 
                                                    for usage scenarios
return RiskScores
\end{lstlisting}


\newpage
\phantomsection
\label{sec:appendix2}
\section*{Приложение Б}
\addcontentsline{toc}{section}{Приложение Б}

Архив с исходными кодами вэб-скрейпера.


\newpage
\phantomsection
\label{sec:appendix3}
\section*{Приложение В}
\addcontentsline{toc}{section}{Приложение В}

Архив с исходными кодами инструмента для разметки датасета.



\newpage
\phantomsection
\label{sec:appendix4}
\section*{Приложение Г}
\addcontentsline{toc}{section}{Приложение Г}

Электронная версия пояснительной записки.


\end{document}