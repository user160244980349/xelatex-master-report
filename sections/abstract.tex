%!TEX root = ../main.tex
\documentclass[../main]{subfiles}

\begin{document}

\newpage
\phantomsection
\section*{Реферат}
Поясн. зап. \total{page} стр., \total{figure}  рис., \total{table} табл., \total{bibcnt} ист., 1 прил.

{   
    \jj\parindent=12.5mm
    \par
    \MakeUppercase{автоматизированная обработка естественного языка, политики безопасности, пользовательские соглашения}
    \par
}

Объектом  исследования являются способы эффективной автоматизированной формализации политик безопасности.

Цель работы -- разработать эффективный план автоматизированных способов формализации политик безопасности на основе онтологического представления, разработать инструменты создания обучающей выборки для автоматизированной формализации политик безопасности.

Политики конфиденциальности предоставляют пользователям информацию о том, как их личные данные собираются, обрабатываются и передаются третьим лицам. Однако в большинстве случаев они написаны нечетко и непрозрачно. Поэтому важно сделать политику конфиденциальности ясной и прозрачной для конечного пользователя. В этой работе исследуется применение методов LSA, LDA, POS для обнаружения семантических тем, представленных в политике конфиденциальности. Также тестируется POS подход с пулами синонимов. Однако такие строгие способы обработки текста не очень точны. Использование методов глубокого обучения с онтологическим представлением предметной области делает возможной точную формализацию политики конфиденциальности. Для этого были созданы поисковый робот и инструмент аннотации. С помощью поисковый бота был получен набор данных из 592 политик конфиденциальности. 

\newpage
\section*{Abstract}
Privacy policies provide end users information about how they personal data collected, processed and shared with third parties. However, in major cases they are written in unclear and not transparent manner. So, it is important to make privacy  policies clear and transparent to end user. In this work, application of the LSA, LDA, POS techniques to detect semantic topics presented in the privacy policy are investigated. Also POS with synonyms pools are tested. However, more strict ways of text processing are not very accurate. Using deep learning techniques with ontology representation of subject field making accurate privacy policy formalization possible. For that the crawler and annotation tool were created. Finally, the privacy policies dataset consisting of 592 was obtained with crawler.

\end{document}