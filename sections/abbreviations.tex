%!TEX root = ../main.tex
\documentclass[../main]{subfiles}

\begin{document}

\newpage
\phantomsection
\addcontentsline{toc}{section}{Перечень сокращений и обозначений}
\section*{Перечень сокращений и обозначений}

В настоящей пояснительной записке к выпускной квалификационной
работе используются следующие сокращения и обозначения.

\begin{termenum}
    \item E-P3P --- (от англ. Platform for Privacy Preferences Project) протокол, позволяющий веб-сайтам заявлять о предполагаемом использовании собираемой информации о пользователях веб-браузера
    \item GPS --- (от англ. Global Positioning System) система глобального позиционирования
    \item IoT --- (от англ. Internet of Things) устройства интернета вещей
    \item LDA --- (от англ. Latent Dirichlet Allocation) латентное размещение Дирихле
    \item LSA --- (от англ. Latent Semantic Search) латентно-семантический анализ
    \item ML --- (от англ. Machine Learning) машинное обучение
    \item NLP --- (от англ. Natural Language Processing) обработка естественного языка
    \item PII --- (от англ. Personally Identifiable Information) информация об идентифицируемом субъекте
    \item POS --- (от англ. Part Of Speech) разложение по частям речи
    \item SVC --- (от англ. Support Vector Machine) метод опорных векторов
    \item TF-IDF --- (от англ. Term Frequency -- Inverse Document Frequency) инверсная частотная характеристика документа
    \item UML --- (от англ. Unified Modeling Language) унифицированный язык моделирования
    \item Wi-Fi --- (от англ. Wireless Fidelity) технология беспроводной локальной сети с устройствами на основе стандартов IEEE 802.11.
    \item СУБД --- Система Управления Базами Данных
\end{termenum}

\end{document}